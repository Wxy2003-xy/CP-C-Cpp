\documentclass{article}
\usepackage{amsmath}   % For advanced mathematical formatting
\usepackage{amssymb}   % For mathematical symbols
\usepackage{geometry}  % Adjust page margins
\usepackage{enumerate} % For custom lists
\usepackage{xcolor}  % for coloring
\usepackage{amsthm}
\newtheorem{theorem}{Theorem}[section]
\newtheorem{lemma}[theorem]{Lemma}
\newtheorem{corollary}[theorem]{Corollary}
\newtheorem{definition}[theorem]{Definition}
\usepackage{listings}  % for code listings

\lstset{frame=tb,
  language=C,
  aboveskip=3mm,
  belowskip=3mm,
  showstringspaces=false,   
  columns=flexible,
  basicstyle={\small\ttfamily},
  numbers=none,
  numberstyle=\tiny\color{gray},
  keywordstyle=\color{blue},
  commentstyle=\color{brown},
  stringstyle=\color{orange},
  breaklines=true,
  breakatwhitespace=true,
  tabsize=3
}
\geometry{top=1in, bottom=1in, left=1in, right=1in}

\begin{document}

\title{Minimum Diameter after merging two trees}
\author{Wang Xiyu}
\date{24 Dec 2024}
\maketitle

\section*{Problem Statement}
Given 2 undirected unweighted trees, find the minimal possible diameter of a new tree formed by connecting the two trees with exactly one new edge.
\section*{Problem Constraints}
Both graphs are trees, meaning both are connected and acyclic.
\section*{Concept Proof}
\begin{theorem}
The diameter of the new merged tree is at least the diameters of both original trees.
\end{theorem}
\begin{proof}
    Suppose not, the diameter of one tree $G$ is $k$, a path of length $k$ between $v_1$ to $v_2$.
    By merging, there exists a new path $v_1 \vdash^* v_2$ that is of length $l$, that $l < k$.
    Now there exist a cycle in $G$, $v_1 \vdash^* v_2 \vdash^* v_1$ that is of length $l + k$. 
    The resulting graph is not a tree, a contradiction.  
\end{proof}
\begin{theorem}
    Given the diameter of tree 1 $k_1$ and diameter of tree $k_2$, where $max(k_1, k_2) \leq 2 \times min(k_1, k_2)$  the minimal diameter of the merged tree is
    $max(k_1, k_2, \lceil \frac{k_1}{2} \rceil + \lceil \frac{k_2}{2} \rceil + 1)$
\end{theorem}
\begin{proof}
    Consider one path between $v$ and $v'$, $v, v' \in V_1$ in tree 1 with diameter $k_1$
    in tree 1, find the center of the path $v''$, $v \vdash^{\lceil \frac{k_1}{2} \rceil} v'' \land v'' \vdash^{\lfloor \frac{k_1}{2} \rfloor} v'$.
    Do the same on the other tree with diameter $k_2$
    A new edge will be added between $v''$ and a node in the other tree, the diameter of the merged tree will become the larger between $\lceil \frac{k_1}{2} \rceil + max(\lfloor \frac{k_1}{2} \rfloor, \lceil \frac{k_2}{2} \rceil + 1)$ and
    $\lceil \frac{k_2}{2} \rceil + max(\lfloor \frac{k_2}{2} \rfloor, \lceil \frac{k_1}{2} \rceil + 1)$, corresponding to the longer by 1 halves of diameter path of both trees, plus the new edge joining them together.
\end{proof}

\begin{theorem}
    The shortest longest path in a tree both starts and ends in a leaf node.
\end{theorem}

\begin{proof}
    Suppose not, a shortest longest path $(v, v')$ ends in a non-leaf node $v'$, $\exists v''\in V, 
    v''$ is a leaf and $v' \vdash^* v''$. By definition of trees, a leaf has at most one parent, $!\exists e \in E, e = (v, v''), len(e) = len((v, v')) + len((v', v''))$. 
    With no negatively weighted edges, $len((v, v'')) > len((v, v')) + len((v', v''))$, $(v, v')$ is not a shortest longest path. Contradiction.
\end{proof}
\begin{lemma}
    The diameter of a tree can be found by performing 2 DFS.
\end{lemma}
\begin{proof}
    Both ends of a shortest longest path are a leaf, and in a tree every other node is reachable from any node via a unique path (trivial). 
    Therefore a longest possible shortest longest path in a tree, is the longest shortest path walkable from one leaf to another leaf.
    By performing one DFS, we can find a leaf in $O(V)$ time, and another DFS starting from the leaf found in the first DFS can find the length of the shortest path from the leaf
    to every other node; among which the longest is the diameter of the tree. 
\end{proof}

\section*{Solution}
\begin{lstlisting}
    class Solution {
        public:
            int minimumDiameterAfterMerge(vector<vector<int>>& edges1,
                                        vector<vector<int>>& edges2) {
                int dia1 = findDiameter(edges1);
                int dia2 = findDiameter(edges2);
                int a = ceil(dia1/2.0);
                int b = ceil(dia2/2.0);
                return max({dia1, dia2, (1 + a + b)});
            }
            int findDiameter(vector<vector<int>>& edges) {
                if (edges.empty()) {
                    return 0;
                }
                int len = edges.size() + 1;
                vector<vector<int>> adj = processAdj(edges, len); // process input to form adjacency list
                if (adj.empty() || (adj.size() == 1 && adj[0].empty())) {
                    return 0;  // Single node or empty adjacency list
                }
                vector<bool> visited(len, false); // initialize visited array
                stack<pair<int, int>> s;
                s.push({0, 0});
                int depth = 0;
                int furthest = 0;
                while (!s.empty()) {
                    pair<int, int> n = s.top();
                    s.pop();
                    int currNode = n.first;
                    int currLevel = n.second;
                    if (!visited[currNode]) {
                        visited[currNode] = true;
                        if (currLevel > depth) {
                            depth = currLevel;
                            furthest = currNode;
                        }
                        for (int i = 0; i < adj[currNode].size(); i++) {
                            s.push(make_pair(adj[currNode][i], currLevel + 1));
                        }
                    }
                }
                vector<bool> visited2(len, false);
                s.push({furthest, 0});
                int diameter = 0;
                while(!s.empty()) {
                    pair<int, int> n = s.top();
                    s.pop();
                    int currNode = n.first;
                    int currLevel = n.second;
                    if (!visited2[currNode]) {
                        visited2[currNode] = true;
                        diameter = max(diameter, currLevel);
                        if (adj[currNode].size() > 0) {
                            for (int i = 0; i < adj[currNode].size(); i++) {
                                s.push(make_pair(adj[currNode][i], currLevel + 1));
                            }
                        }
                    }
                }
                return diameter;
            }
            vector<vector<int>> processAdj(vector<vector<int>>& edges, int len) {
                vector<vector<int>> adj(len);
                if (len == 1) {
                    return adj;
                }
                for (int i = 0; i < edges.size(); i++) {
                    adj[edges[i][0]].push_back(edges[i][1]);
                    adj[edges[i][1]].push_back(edges[i][0]);
                }
                return adj;
            }
};
\end{lstlisting}

\end{document}
